
\chapter{Related Literature and Studies}
\begin{refsection}

This chapter provides a comprehensive overview of the literature and studies related to spatiotemporal landmark recognition and localization, focusing on \gls{ar}-based campus navigation. It also synthesizes critical similarities and differences among existing research on \gls{cnn} and \gls{lstm}. Lastly, the chapter identifies a gap in the current literature and discusses how the present study aims to address this gap.

\section{Review of Related Literature and Studies}

\subsection{Spatiotemporal Landmark Recognition and Localization}

Today, modern deep learning techniques are making rapid progress in the field of spatiotemporal landmark recognition and localization. Spatial feature recognition previously required manual labeling, which is quite labor intensive and very inefficient \cite{three}; deep learning, according to recent work, particularly that of \citeauthor{three} \citeyear{three}, has actually improved recognition performance by better elucidation of spatial relationships between objects. 

In addition, \citeauthor{six} \citeyear{six} showcased pre-trained \gls{cnn} models that were effective in extracting spatial features from hyperspectral images for enhanced classification performance, thus establishing the value of transfer learning for improved performance in different image classification tasks bordering on complex geospatial data \cite{six}. The adoption of \gls{ml} in clustering and recognizing spatial data patterns has further widened the horizon of geospatial analysis. \citeauthor{seven} \citeyear{seven} commented that this would help the automatic extraction of considerable patterns from complex datasets, which is very crucial for processing geographical data, such as satellite imagery, and unraveling hidden trends that would aid decision-making \cite{seven}.

In addition, \citeauthor{eight} \citeyear{eight} examined a high-dimensional self-attention mechanism that combines spatial and temporal features to incrementally improve the predictive power behind systematic evaluations with varying neural network architectures. This work stands as an illustration of the growing finesse in the handling of spatiotemporal data \cite{eight}.

The excessive growth of images, especially in web and mobile applications, poses threats to efficient landmark recognition. Traditional methods, such as \gls{svm}, revealed their limitations when handling variations in elevation and structure. Contemporary studies, including \citeauthor{nine} \citeyear{nine}, show that \gls{cnn} architectures such as ResNet-50 have proven successful in achieving high accuracy for the detection of landmarks in various view angles. These advancements justify the abilities of deep learning models, which are helpful in real-world applications, including navigation to identify unlabeled historical landmarks \cite{nine}.

This concludes that the changeover of the spatiotemporal landmark recognition and localization processes into deep learning is reshaping the domain, addressing issues of efficiency and accuracy. Continuous research will positively impact applications in the field with implications that center on advanced capabilities in understanding and articulating spatial environments.

\subsection{\gls{ar}-based Campus Navigation}

Navigating large university campuses can be challenging, especially for new students and visitors unfamiliar with the environment. \gls{ar}-based navigation systems provide an interactive solution, improving indoor and outdoor way-finding by overlaying digital content onto the real world, improving localization accuracy, and offering real-time guidance.  Several studies highlight the practical benefits of \gls{ar} in campus navigation. \citeauthor{one} \citeyear{one} proposed an ARCore-based system that uses visual-inertial ranging and Unity3D to improve positioning in \gls{gps}-limited areas \cite{one}. Similarly, \citeauthor{ten} \citeyear{ten} integrated IoT-based sensor fusion for adaptive route guidance, while \citeauthor{two} \citeyear{two} combined \gls{gps}, Wi-Fi, BLE beacons and \gls{ai}-driven path optimization for voice-assisted, accessible navigation \cite{ten, two}. \citeauthor{thirteens} \citeyear{thirteens} explored \gls{ai} integration through \gls{cnn} and LSTM to improve the precision of localization \cite{thirteens}. Hybrid systems were also developed, such as the \gls{ar}-VR platform by \citeauthor{four} \citeyear{four}, which allows real-time tracking, virtual campus tours and personalized routes, and the landmark-based system by \citeauthor{twelve} \citeyear{twelve}, which improves spatial learning, especially for older adults \cite{four, twelve}.

In addition to supporting indoor-outdoor navigation, \citeauthor{five} \citeyear{five} introduced an \gls{ar} app utilizing computer vision and object detection, while \citeauthor{eleven} \citeyear{eleven} and \citeauthor{fifteen} \citeyear{fifteen} focused on indoor systems that integrate sensor data and \gls{ar} overlays for improved accuracy and user experience \cite{ten, eleven, fifteen}. \citeauthor{fourteen} \citeyear{fourteen} also emphasized the broader developments in \gls{ar}-assisted indoor positioning and mapping technologies \cite{fourteen}.

In conclusion, these studies collectively underscore the potential of \gls{ar}-powered campus navigation systems in delivering accurate, accessible, and interactive guidance. With the integration of real-time tracking, \gls{ai}, and immersive interfaces, \gls{ar} systems continue to enhance the overall wayfinding experience in educational environments.

\subsection{Convolutional Neural Networks}

The integration of \gls{cnn} into \gls{ar}-based navigation systems has significantly enhanced localization accuracy, object recognition, and real-time spatial understanding, making navigation more intelligent and responsive.

Several studies have demonstrated the value of \gls{cnn}s in improving \gls{ar} navigation. \citeauthor{twentytwo} \citeyear{twentytwo} emphasized how \gls{cnn}s enhance scene recognition and real-time interaction in \gls{ar} environments, while \citeauthor{twenty} \citeyear{twenty} developed a \gls{cnn}-augmented SLAM system using planar constraints for more stable positioning \cite{twentytwo, twenty}. Similarly, \citeauthor{seventeen} \citeyear{seventeen} applied \gls{cnn}s to mobile robot localization in indoor spaces, and \citeauthor{twentyfive}\citeyear{twentyfive} extended this by applying deep \gls{cnn}s to dynamic pathfinding in autonomous robot navigation \cite{seventeen, twentyfive}.

Other researchers have focused on \gls{cnn}s to support accessibility. \citeauthor{sixteen} \citeyear{sixteen} implemented \gls{cnn}s in \gls{ar} systems to assist visually impaired users by improving obstacle detection, while \citeauthor{eighteen} \citeyear{eighteen} enhanced environmental perception through \gls{cnn}-powered object recognition and tracking \cite{sixteen, eighteen}. In terms of interaction, \citeauthor{twentyone} \citeyear{twentyone} introduced a \gls{cnn}-based multi-target classification model for \gls{ar}-SSVEP to improve interaction precision \cite{twentyone}.

Broader reviews and advanced implementations further showcase \gls{cnn}'s potential. \citeauthor{nineteen} \citeyear{nineteen} conducted a systematic review of \gls{cnn} use across \gls{ar}, VR, and MR, emphasizing their role in spatial computing and immersive navigation \cite{nineteen}. \citeauthor{twentyfour} \citeyear{twentyfour} combined \gls{cnn}s with LSTM in Kalman filter fusion to refine path correction and trajectory prediction, while Shin-\citeauthor{twentythree} \citeyear{twentythree} demonstrated \gls{cnn}s’ value in surgical \gls{ar} navigation requiring high-precision localization \cite{twentyfour, twentythree}.

In conclusion, these studies collectively highlight the critical role of \gls{cnn}s in advancing \gls{ar}-based navigation. By enabling accurate object recognition, adaptive route planning, and robust spatial understanding, \gls{cnn}s contribute significantly to making \gls{ar} navigation systems more intelligent, accessible, and immersive.

\subsection{Long Short-Term Memory}

State-of-the-art positioning technology has traditionally produced large-scale trajectory data that plays a major role in location prediction for location-based services (LBS). Displacement forecast through long-term visitations following the traditional sense or real-time trajectory forecasting are the mainstream methods. An emerging approach has interest in integrating both the spatial and temporal dimensions together using \gls{lstm} Networks, where, even more specifically, the Spatial-Temporal Long Short-Term Memory (ST-LSTM) model tackles data sparsity and caters to better urban mobility and personalized LBS applications \cite{twentysix}. The SPATIAL architecture provides yet another example of such integration with improved prediction accuracy \cite{twentyseven}.

It is noteworthy that Wi-Fi-based positioning has gained some traction, especially in the indoor setting, but with its challenges in accuracy due to varying signal conditions. A recent spatial-temporal positioning algorithm adopts a residual network for the extraction of spatial features and LSTM, greatly increasing the precision of localization \cite{twentyeight}.

LSTMs may also find utilization in environmental monitoring, thus performing weather prediction and applicable scenarios better than traditional \gls{cnn} and LSTM methods with a hybrid model among variates \cite{twentynine}.

Another area is traffic management. According to \citeauthor{thirty} \citeyear{thirty}, LSTM models outperformed classical methods in estimating traffic speed \cite{thirty}. \citeauthor{thirtyone} \citeyear{thirtyone} demonstrated the successful application of the improved Bi-LSTM model to real-time traffic flow forecasting, further stating that spatial-temporal modeling is crucial in urban traffic systems \cite{thirtyone}. \citeauthor{thirtytwo} \citeyear{thirtytwo} improved trajectory prediction of vehicles by using spatial and temporal attention mechanisms in the STAM-LSTM, capable of capturing the relationships between vehicle and associated motion features \cite{thirtytwo}.

Apart from transportation, \gls{lstm} networks have made strides in facial expression recognition; the Enhanced ConvLSTM model leverages the spatial and temporal connections and is efficient in complex environments \cite{thirtythree}. In a parallel manner, the STGA-LSTM framework by \citeauthor{thirtyfour} predicts short-term demand for bike-sharing using Graph Convolutional Networks, witnessing an unprecedented advancement in demand forecasting \cite{thirtyfour}. Then, in predicting outlet temperature for energy systems, a hybrid model \gls{cnn}-LSTM was established on modeling spatial-temporal features for better thermal energy management \cite{thirtyfive}, proposed by \citeauthor{thirtyfive} \citeyear{thirtyfive}.

This coupling of spatial and temporal factors in LSTMs has led to the elimination of boundaries in predictive analytics across various fields. This development indicates that the model may be the answer to stochastic problems posed in real-life scenarios, such as intelligent transportation systems, environmental monitoring, urban mobility, and energy management. LSTM has also begun to set the way for future breakthroughs in predictive modeling and analytics as research continues.

\subsection{Evaluation Metrics and Performance Analysis}

Evaluating the effectiveness of AR-based campus navigation systems requires a comprehensive assessment of their positional accuracy. Different metrics serve to illuminate various aspects of system performance. For large deployable mesh reflectors, \citeauthor{thirtyseven} \citeyear{thirtyseven} highlights the usefulness of the root-mean-square (RMS) error as an effective surface accuracy measure. RMS calculates the average deviation between the actual surface and the intended shape, providing a single value that summarizes overall surface fidelity. Yuan's study compares several RMS measurement approaches: nodal deviation offers highly localized details but demands extensive computational effort; the best-fit surface method provides a balanced, global view of shape accuracy; and direct RMS error yields a simple overall deviation metric, though it may miss localized imperfections. Incorporating assessments focused on critical surface regions enhances the evaluation by highlighting areas most vital to system performance \cite{thirtysix}. 

In the context of indoor positioning, \citeauthor{thirtyseven} \citeyear{thirtyseven} demonstrate the importance of path-based error metrics like the Mean Euclidean Error (MEE). Their research compares different methods and finds that the Visibility Graph (VG) approach achieves the lowest average error of about 2.2 meters, closely reflecting the actual routes pedestrians walk in complex indoor spaces. The Navigation Mesh (NM) method follows with an error near 2.4 meters, while the Fast Marching (FM) method results in a higher average error of approximately 3.7 meters. These findings emphasize that MEE, which measures the length of paths pedestrians would realistically follow, provides a more practical, real-world assessment of localization accuracy than simple straight-line distances. Since MEE closely relates to how users navigate environments, it offers valuable insights into system usability and reliability.  \cite{thirtyseven}. 

The evaluation results for navigation time in the study of \citeauthor{one} \citeyear{one}, indicates that during the experiments, the AR navigation system achieved approximately a 20\% reduction in navigation times compared to traditional maps at Shanghai University, which is a notable improvement in efficiency. The mention of "p < 0.05" refers to the p-value obtained from statistical testing, likely a t-test or similar, which measures the probability that the observed difference occurred by chance. A p-value less than 0.05 is commonly considered statistically significant, meaning there is less than a 5\% probability that the observed reduction in navigation time was due to random variation alone. This significance level provides confidence that the AR system genuinely enhances navigation speed, validating its effectiveness. The combination of practical results (20\% reduction) and statistical validation (p < 0.05) supports the conclusion that the AR-based system offers a meaningful and reliable improvement for campus navigation \cite{one}. 


\section{Synthesis of the State of the Art}

The evolution of spatiotemporal landmark recognition and localization has been significantly shaped by deep learning techniques, particularly with the integration of \gls{cnn} and \gls{lstm} networks. Traditional spatial recognition methods, which relied heavily on manual landmark identification, have proven inefficient and time-consuming \cite{three}. The work of \citeauthor{three} \citeyear{three} emphasized the use of deep learning in enhancing recognition accuracy and spatial relationship modeling \cite{three}. \citeauthor{six} \citeyear{six} highlighted the benefit of transfer learning with pre-trained \gls{cnn}s in improving classification accuracy, especially in hyperspectral imagery \cite{six}. Similarly, \citeauthor{seven} \citeyear{seven} noted that \gls{ml} improved spatial clustering and pattern recognition, which are vital in geospatial analysis \cite{seven}.

To further boost prediction accuracy, \citeauthor{eight} \citeyear{eight} proposed a high-dimensional self-attention mechanism for fusing spatial and temporal features \cite{eight}. In contrast, \citeauthor{nine}\citeyear{nine} pointed out the limitations of traditional algorithms like Support Vector Machines (SVMs), especially when dealing with orientation variations, where \gls{cnn} models such as ResNet-50 performed better in landmark detection \cite{nine}. These advancements have practical applications in navigation, automated mapping, and landmark identification across diverse environments.

\gls{ar}-based campus navigation systems have emerged as practical applications of these models, offering interactive and real-time wayfinding solutions. \citeauthor{one} \citeyear{one} designed an ARCore-based system utilizing visual-inertial ranging algorithms for enhanced positioning, while \citeauthor{ten} \citeyear{ten} incorporated IoT-based sensor fusion to enable adaptive route guidance \cite{one, ten}. Similarly, \citeauthor{two} \citeyear{two} integrated \gls{gps}, Wi-Fi triangulation, BLE beacons, and \gls{ai}-driven path optimization to support voice-assisted and event-based routing \cite{two}. \citeauthor{thirteens} \citeyear{thirteens} focused on \gls{ai}-enhanced \gls{ar} systems using \gls{cnn} and LSTM models to improve navigation accuracy. \citeauthor{four} \citeyear{four} developed a hybrid \gls{ar}-VR system that supports virtual campus exploration and personalized routing \cite{thirteens, four}.

\citeauthor{twelve} \citeyear{twelve} developed a landmark-based navigation system targeting older adults, while \citeauthor{fourteen} \citeyear{fourteen} provided a broad overview of \gls{ar}-assisted localization techniques \cite{twelve, fourteen}. \citeauthor{five} \citeyear{five} emphasized the use of computer vision and object detection, and \citeauthor{eleven} \citeyear{eleven}focused on enhancing indoor navigation through sensor integration \cite{five, eleven}. Lastly, \citeauthor{fifteen} \citeyear{fifteen} streamlined campus wayfinding by overlaying directional markers in \gls{ar} environments. Together, these works highlight the growing impact of \gls{ar} in educational settings for real-time and user-friendly navigation \cite{fifteen}.

\gls{cnn} integration continues to be a major force behind advancements in AR. \citeauthor{twentytwo} \citeyear{twentytwo} and \citeauthor{twenty} \citeyear{twenty} showed how \gls{cnn}s enhance real-time scene recognition and positioning \cite{twentytwo, twenty}. \citeauthor{sixteen} \citeyear{sixteen} and \citeauthor{seventeen} \citeyear{seventeen} applied \gls{cnn}s to assist visually impaired users and optimize indoor navigation \cite{sixteen, seventeen}. \citeauthor{twentyfive} \citeyear{twentyfive}, \citeauthor{eighteen} \citeyear{eighteen}, and \citeauthor{twentyone} \citeyear{twentyone} further validated \gls{cnn}s’ capacity for object recognition and interaction in \gls{ar} contexts \cite{twentyfive, eighteen, twentyone}. \citeauthor{nineteen} \citeyear{nineteen} reviewed \gls{cnn} applications in AR, VR, and MR, while \citeauthor{twentyfour} \citeyear{twentyfour} and \citeauthor{twentythree} \citeyear{twentythree} demonstrated \gls{cnn}s' effectiveness in surgical and predictive navigation systems \cite{nineteen, twentyfour, twentythree}.

\gls{lstm} networks also play an important role by capturing spatial-temporal dependencies in path prediction and localization. \citeauthor{twentysix} \citeyear{twentysix}, \citeauthor{twentyseven} \citeyear{twentyseven}, \citeauthor{twentyeight} \citeyear{twentyeight}, all demonstrated the application of \gls{lstm} networks to improve accuracy in various domains, from location tracking to environmental forecasting \cite{twentysix, twentyseven, twentyeight}. \citeauthor{twentynine} \citeyear{twentynine} and \citeauthor{thirty} \citeyear{thirty} showed \gls{lstm} networks' predictive accuracy in temperature modeling and traffic speed forecasting \cite{twentynine, thirty}. \citeauthor{thirtyone} \citeyear{thirtyone} and \citeauthor{thirtytwo} \citeyear{thirtytwo} focused on \gls{lstm} variants such as Bi-LSTM and STAM-LSTM for vehicle trajectory predictions, while \citeauthor{thirtythree} \citeyear{thirtythree}, \citeauthor{thirtyfour} \citeyear{thirtyfour}, and \citeauthor{thirtyfive} \citeyear{thirtyfive} explored their integration with \gls{cnn} and GCN for specific applications such as emotion recognition and energy modeling \cite{thirtyone, thirtytwo, thirtythree, thirtyfour, thirtyfive}.

In evaluating \gls{ar}-based campus navigation systems, several key metrics are utilized to assess efficiency, accuracy, and reliability. Root Mean Squared Error (RMSE) helps capture significant localization discrepancies by emphasizing larger errors, as supported by \citeauthor{thirtysix} \citeyear{thirtysix}, who demonstrated that RMSE effectively assesses surface accuracy, with methods such as the best-fit surface that offers balanced global evaluations \cite{thirtysix}. Mean Euclidean Error (MEE), according to \citeauthor{thirtyseven} \citeyear{thirtyseven}, provides realistic information on positioning performance by measuring the average distance between predicted and actual paths, and their study shows that VG achieved lower MEE values around ~2.2 meters, highlighting its importance in dynamic real-world environments \cite{thirtyseven}. Navigation time, evaluated in the study by \citeauthor{one} \citeyear{one}, showed that \gls{ar}-based systems reduced navigation time by approximately 20\% compared to traditional map applications, significantly improving user navigation speed and efficiency in campus environments \cite{one}. Together, these metrics; RMSE, MEE, and navigation time offer a comprehensive evaluation of \gls{ar} navigation systems, guiding future improvements to improve precision, operational speed, and overall user experience.

In summary, deep learning and \gls{ar} technologies have converged to significantly improve spatial navigation. \gls{cnn}s and \gls{lstm} networks excel in localization, prediction, and interaction modeling, while the integration of RMSE, MEE, and Navigation Time ensures the continuous advancement of more accurate and user-centric navigation systems.

\section{Gap Bridged by the Study}

Existing studies have shown the success and potential of spatial models for \gls{ar} navigation and positioning. However, there is still room for improvement, especially since many current \gls{ar} navigation systems rely on \gls{gps} which uses internet connectivity and focuses mainly on indoor navigation. This creates a gap in the provision of effective outdoor navigation solutions, particularly in large campus areas where reliable guidance is crucial.

This study aims to address this gap by developing an offline spatiotemporal localization model that uses \gls{cnn} and \gls{lstm} networks. The main goal is to create an interactive way-finding solution for \gls{cspc}.

%=======================================================%
%%%%% Do not delete this part %%%%%%
\clearpage

\printbibliography[heading=subbibintoc, title={\centering Notes}]
\end{refsection}